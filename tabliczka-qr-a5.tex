%%%%%%%%%%%%%%%%%%%%%%%%%%%%%%%%%%%%%%%%%
%                                       %
%  TABLICZKA NA DRZWI                   %
%  wg księgi indentyfikacji wizualnej   %
%  Politechniki Śląskiej                %
%                                       %
%  (c) Krzysztof Simiński, 2020-2023    %
%                                       %
%%%%%%%%%%%%%%%%%%%%%%%%%%%%%%%%%%%%%%%%%

% UWAGI
% Konieczne jest zainstalowanie fontu Barlow Semi Condensed.
% kompilacja: xelatex

\documentclass{standalone}

\usepackage{fontspec}
\defaultfontfeatures{Mapping=tex--text}   % to support TeX conventions like dashes etc
\usepackage{xltxtra} % extra customisation for XeLaTeX
\setsansfont{Barlow Semi Condensed}

\usepackage{qrcode}
\usepackage{ifthen}
 
\usepackage[polish]{babel} 
\usepackage{graphicx}
\usepackage{tikz}
\usepackage{xcolor}


% ustawienie kolorów
\definecolor{polsl-niebieski}{RGB}{46, 79, 158}
%\definecolor{polsl-niebieski}{rgb}{0.00000, 0.29411 , 0.56862}
%\definecolor{polsl-niebieski}{rgb}{0.18039, 0.30980 , 0.61961}
\definecolor{polsl-zolty}    {rgb}{1.00000, 0.84314 , 0.00000}


\newcommand{\HUGE}{\fontsize{35}{30}\selectfont}
\newcommand{\veryHuge}{\fontsize{65}{55}\selectfont}

\frenchspacing

% parametry bloku opisisującego osobę:
% pierwszy opcjonalny: długość kreski pod imienim i nazwiskiem (domyślnie 8.5cm)
% współrzędna pionowa bloku (im wyższa wartość, tym wyższe położenie)
% stopień / tytuł
% imię nazwisko
% stanowisko
% kod QR (może być puste)
\newcommand{\osoba}[7][8.5cm]{%
\ifthenelse{\equal{#7}{}}{%empty
}
{%nonempty
\draw (64mm, #2 + 5mm) node {\qrcode[height=2cm]{http://www.polsl.pl}};
}
\draw (80mm, #2 + 10mm) node [anchor=south west] {\sffamily\Large\vphantom{Iy} #3};
\draw (80mm, #2) node[anchor=south west] {\sffamily\bfseries\Huge\vphantom{Iy} #4 \MakeUppercase{#5}};
\draw[fill] (81.5mm, #2 - 0mm) rectangle +(#1, 0.5mm);
\draw (80mm, #2 - 7mm) node [anchor=south west] {\sffamily\vphantom{Iy} #6};
} 
 
\begin{document} 
\begin{tikzpicture}
\draw[] (0,0) rectangle (210mm, 148mm);
\draw[polsl-niebieski,fill=polsl-niebieski] (0,0) rectangle (45mm, 148mm);
\draw[polsl-zolty,fill=polsl-zolty] (0mm, 125mm) rectangle +(52mm, 2mm);


% logo 
% tylko PolSl
%\node[inner sep=0pt] (logo) at (22.5mm, 3cm) {\includegraphics[width=35mm]{graf/politechnika_sl_logo_pion_inwersja_pl.eps}};

% PolSl + uczelnia badawcza
\node[inner sep=0pt] (logo) at (13.0mm, 2.6cm) {\includegraphics[width=24mm]{graf/politechnika_sl_logo_pion_inwersja_pl.eps}};
\node[inner sep=0pt] (logo) at (32.0mm, 2.4cm) {\includegraphics[width=15mm]{graf/uczelnia-badawcza-kolor-pion.png}};



% numer pokoju
\draw (60mm, 125mm) node[anchor=south west] {\sffamily\bfseries\veryHuge 123a};


% osoby
\osoba[110mm]%   opcjonalna długość kreski pod nazwiskiem, wartość domyślna 8.5 cm
{105mm}%         położenie w pionie
{dr hab. inż.}%  tytulatura
{Imię}%          imię :-)
{Nazwisko}%      nazwisko
{profesor Politechniki Śląskiej / professor of The Silesian University of Technology} % stanowisko
{www.polsl.pl}%              kod QR    
;

\osoba[60mm]{75mm}{dr inż.}{Imię}{Nazwisko}{adiunkt / assisting professor}{};
\osoba[60mm]{45mm}{mgr inż.}{Imię}{Nazwisko}{asystent / assistent}{};


% wydział
\draw[polsl-niebieski] (60mm, 2.2cm) node [anchor=south west,text width=13cm,align=left] {\sffamily\Large Wydział po polsku\\Faculty in English};

% katedra
\draw[polsl-niebieski] (60mm, 1.0cm) node [anchor=south west,text width=13cm,align=left] {\sffamily\Large Katedra po polsku\\Chair in English};

\end{tikzpicture} 
\end{document}